% chktex-file 44
\section{Port Descriptions}

\subsection{GPIO Interface}

The ports for \textbf{GPIO} are shown below in 
Table 1. The width of several ports is controlled 
by the following input parameters:

\textit{dataWidth} is the width of the gpioInput, gpioOutput, and gpioOutputEnable ports in bits
 
\renewcommand*{\arraystretch}{1.4}
\begin{longtable}[H]{
  | p{0.20\textwidth}
  | p{0.20\textwidth}
  | p{0.12\textwidth}
  | p{0.43\textwidth} |
  }
  \hline
  \textbf{Port Name} &   
  \textbf{Width} &   
  \textbf{Direction} &   
  \textbf{Description} \\ \hline \hline

  gpioInput &       
  \textit{dataWidth} & 
  Input &       
  Data to be sent to the GPIO\\ \hline

  gpioOutput &        
  \textit{dataWidth} & 
  Output &       
  Data to be recieved from the GPIO \\ \hline

  gpioOutputEnable &      
  \textit{dataWidth} & 
  Output &     
  Enable data to be recieved from the GPIO \\ \hline
 
  \caption{GPIO Ports Descriptions}\label{table:ports}
\end{longtable}

\subsection{APB3 Interface}
The \textbf{APB3 Interface} is a regular APB3 Slave Interface. All signals supported are shown below in 
Table 2. See the \textit{AMBA APB Protocol Specifications} for a complete description of the signals. The width of several ports is controlled 
by the following input parameters:

\begin{itemize}[noitemsep]
  \item \textit{PDATA\_WIDTH} is the width of PWDATA and PRDATA in bits
  \item \textit{PADDR\_WIDTH} is the width of PADDR in bits
\end{itemize}
 
\renewcommand*{\arraystretch}{1.4}
\begin{longtable}[H]{
  | p{0.20\textwidth}
  | p{0.20\textwidth}
  | p{0.12\textwidth}
  | p{0.43\textwidth} |
  }
  \hline
  \textbf{Port Name} &   
  \textbf{Width} &   
  \textbf{Direction} &   
  \textbf{Description} \\ \hline \hline

  PCLK &       
  1 &       
  Input &       
  Positive edge clock \\ \hline

  PRESETN &       
  1 &       
  Input &       
  Active low reset \\ \hline

  PSEL &       
  1 & 
  Input &       
  Indicates slave is selected and a data transfer is required \\ \hline

  PENABLE &        
  1 & 
  Input &       
  Indicates second cycle of APB transfer \\ \hline

  PWRITE &        
  1 & 
  Input &       
  Indicates write access when HIGH and read access when LOW\\ \hline

  PADDR &      
  \textit{PADDR\_WIDTH} & 
  Input &     
  Address bus \\ \hline

  PWDATA &      
  \textit{PDATA\_WIDTH} & 
  Input &     
  Write data bus driven when PWRITE is HIGH\\ \hline

  PRDATA &      
  \textit{PDATA\_WIDTH} & 
  Output &     
  Read data bus driven when PWRITE is LOW\\ \hline
 
  PREADY &        
  1 & 
  Output &       
  Transfer ready \\ \hline

  PSLVERR &        
  1 & 
  Output &       
  Transfer error \\ \hline

  \caption{APB Ports Descriptions}\label{table:interface}
\end{longtable}

