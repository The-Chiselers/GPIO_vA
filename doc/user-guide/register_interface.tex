% chktex-file 44
\section{Register Interface}
 
When programming registers, each register starts on a byte address, and the last bits it would take up in its final byte based on its size are unused. To find the size in bytes for any register, divide by the register size, and round up to the nearest whole number. For example, a 32-bit register would take up 4 bytes, and a 1-bit register would take up 1 byte.
 
\begin{longtable}[H]{
  | p{0.27\textwidth}
  | p{0.18\textwidth}
  | p{0.50\textwidth} |
  }
  \hline
  \textbf{Name} &   
  \textbf{Size (Bits)} &   
  \textbf{Description} \\ \hline \hline

  DIRECTION  &   
  dataWidth &   
  DESC TODO \\ \hline

  OUTPUT &   
  dataWidth &   
  DESC TODO \\ \hline

  INPUT &   
  dataWidth &   
  DESC TODO \\ \hline

  MODE &   
  dataWidth &   
  DESC TODO \\ \hline

  ATOMIC\_OPERATION &   
  4 &   
  DESC TODO \\ \hline

  ATOMIC\_MASK &   
  p.dataWidth &   
  DESC TODO \\ \hline

  ATOMIC\_SET &   
  1 &   
  DESC TODO \\ \hline

  VIRTUAL\_PORT\_MAP &   
  sizeOfVirtualPorts &   
  DESC TODO \\ \hline

  VIRTUAL\_PORT\_OUTPUT &   
  numVirtualPorts &   
  DESC TODO \\ \hline

  VIRTUAL\_PORT\_ENABLE &   
  1 &   
  DESC TODO \\ \hline

  TRIGGER\_TYPE &   
  dataWidth &   
  DESC TODO \\ \hline
  
  TRIGGER\_LO &   
  dataWidth &   
  DESC TODO \\ \hline
  
  TRIGGER\_HI &   
  dataWidth &   
  DESC TODO \\ \hline
  
  TRIGGER\_STATUS &   
  dataWidth &   
  DESC TODO \\ \hline
  
  IRQ\_ENABLE &   
  dataWidth &   
  DESC TODO \\ \hline

\end{longtable}

  \newpage

  \subsection{DIRECTION}
  DIRECTION is a \textit{dataWidth} bits wide active-high read/write register. This register controls the 
  output enable bus \textit{gpioOutputEnable}. Operation can be seen in Table
  \begin{table}[h]
    \centering
    \begin{tabular}{|c|c|}
        \hline
        \textbf{DIRECTION[n]} & \textbf{Direction} \\ \hline
        0 & Input \\ \hline
        1 & Output \\ \hline
    \end{tabular}
    \caption{DIRECTION Register}
\end{table}

\subsection{OUTPUT}
OUTPUT is a \textit{dataWidth} bits wide read/write register. This register controls the 
output bus \textit{gpioOutput}. Writing a '0' drives the pad low in both modes of operation. Writing a '1' 
drives the pad high in push-pull mode, or Hi-Z in open-drain mode. 

\subsection{INPUT}
INPUT is a \textit{dataWidth} bits wide read-only register. This register is written to from the 
input bus \textit{gpioInput}. On the rising edge of the APB3 Bus Clock (PCLK), input data from \textit{gpioInput} 
is synchronized using two flops and stored in the INPUT register. From there, it may be read via the APB3 Bus Interface
through PRDATA. 

\subsection{MODE}
MODE is a \textit{dataWidth} bits wide read/write register. This register sets the operating mode for each bit
of the \textit{gpioOutput} and \textit{gpioOutputEnable} busses as either push-pull or open drain mode. Operation can be seen in Table
\begin{table}[h]
  \centering
  \begin{tabular}{|c|c|}
      \hline
      \textbf{MODE[n]} & \textbf{Operating Mode} \\ \hline
      0 & Push-Pull \\ \hline
      1 & Open Drain\\ \hline
  \end{tabular}
  \caption{MODE Register}
\end{table}

\subsection{TRIGGER\_TYPE}
TRIGGER\_TYPE is a \textit{dataWidth} bits wide read/write register. This register configures whether \textit{gpioInput} is a level or edge sensitive
interrupt trigger as seen below:
\begin{table}[h]
  \centering
  \begin{tabular}{|c|c|}
      \hline
      \textbf{TRIGGER\_TYPE[n]} & \textbf{Type} \\ \hline
      0 & Level \\ \hline
      1 & Edge \\ \hline
  \end{tabular}
  \caption{TRIGGER\_TYPE Register}
\end{table}

\subsection{TRIGGER\_LO}
TRIGGER\_LO is a \textit{dataWidth} bits wide read/write register. This register configures whether the interrupt is triggered on a level low, or a falling edge, 
of \textit{gpioInput} depending on how TRIGGER\_TYPE is set. Operation can be see in Table:
\begin{table}[h]
  \centering
  \begin{tabular}{|c|c|c|}
      \hline
      \textbf{TRIGGER\_LO[n]} & \textbf{Level Trigger} & \textbf{Edge Trigger} \\ \hline
      0 & No Trigger when Low & No Trigger on Falling Edge\\ \hline
      1 & Trigger when Low & Trigger on Falling Edge\\ \hline
  \end{tabular}
  \caption{TRIGGER\_LO Register}
\end{table}

\subsection{TRIGGER\_HI}
TRIGGER\_HI is a \textit{dataWidth} bits wide read/write register. This register configures whether the interrupt is triggered on a level high, or a rising edge, 
of \textit{gpioInput} depending on how TRIGGER\_TYPE is set. Operation can be see in Table:
\begin{table}[h]
  \centering
  \begin{tabular}{|c|c|c|}
      \hline
      \textbf{TRIGGER\_HI[n]} & \textbf{Level Trigger} & \textbf{Edge Trigger} \\ \hline
      0 & No Trigger when High & No Trigger on Rising Edge\\ \hline
      1 & Trigger when High & Trigger on Rising Edge\\ \hline
  \end{tabular}
  \caption{TRIGGER\_HI Register}
\end{table}

\subsection{TRIGGER\_STATUS}
TRIGGER\_STATUS is a \textit{dataWidth} bits wide read/write register. This register sets a corresponding bit to '1' if a trigger condition is met on
the corresponding \textit{gpioInput[n]} according to the settings of TRIGGER\_TYPE, TRIGGER\_LO, and TRIGGER\_HI.
\newline
\newline
TRIGGER\_STATUS may be read on the PRDATA bus to determine if a trigger condition has occurred. Writing a '1' to TRIGGER\_STATUS[n] will clear the 
status of the corresponding bit. If a new trigger is detected simulatenously during this write, the TRIGGER\_STATUS[n] will remain set.
\begin{table}[h]
  \centering
  \begin{tabular}{|c|c|}
      \hline
      \textbf{TRIGGER\_STATUS[n]} & \textbf{Status} \\ \hline
      0 & No Trigger Detected \\ \hline
      1 & Trigger Detected\\ \hline
  \end{tabular}
  \caption{TRIGGER\_STATUS Register}
\end{table}

\newpage
\subsection{IRQ\_ENABLE}
IRQ\_ENABLE is a \textit{dataWidth} bits wide read/write register. This register determines if the \textit{irqOutput} pin is asserted when a trigger condition
occurs on the corresponding TRIGGER\_STATUS[n]. IRQ\_ENABLE is responsible for enabling interrupt generation from the GPIO core.
\begin{table}[h]
  \centering
  \begin{tabular}{|c|c|}
      \hline
      \textbf{IRQ\_ENABLE[n]} & \textbf{Definition} \\ \hline
      0 & Disable IRQ Generation\\ \hline
      1 & Enable IRQ Generation \\ \hline
  \end{tabular}
  \caption{IRQ\_ENABLE Register}
\end{table}



