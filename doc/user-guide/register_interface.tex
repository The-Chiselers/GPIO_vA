% chktex-file 44
\section{Register Interface}
 
When programming registers, each register starts on a byte address, and the last bits it would take up in its final byte based on its size are unused. To find the size in bytes for any register, divide by the register size, and round up to the nearest whole number. For example, a 32-bit register would take up 4 bytes, and a 1-bit register would take up 1 byte.
 
\begin{longtable}[H]{
  | p{0.27\textwidth}
  | p{0.18\textwidth}
  | p{0.50\textwidth} |
  }
  \hline
  \textbf{Name} &   
  \textbf{Size (Bits)} &   
  \textbf{Description} \\ \hline \hline

  DIRECTION  &   
  dataWidth &   
  DESC TODO \\ \hline

  OUTPUT &   
  dataWidth &   
  DESC TODO \\ \hline

  INPUT &   
  dataWidth &   
  DESC TODO \\ \hline

  MODE &   
  dataWidth &   
  DESC TODO \\ \hline

  ATOMIC\_OPERATION &   
  4 &   
  DESC TODO \\ \hline

  ATOMIC\_MASK &   
  p.dataWidth &   
  DESC TODO \\ \hline

  ATOMIC\_SET &   
  1 &   
  DESC TODO \\ \hline

  VIRTUAL\_PORT\_MAP &   
  sizeOfVirtualPorts &   
  DESC TODO \\ \hline

  VIRTUAL\_PORT\_OUTPUT &   
  numVirtualPorts &   
  DESC TODO \\ \hline

  VIRTUAL\_PORT\_ENABLE &   
  1 &   
  DESC TODO \\ \hline

  TRIGGER\_TYPE &   
  dataWidth &   
  DESC TODO \\ \hline
  
  TRIGGER\_LVL0 &   
  dataWidth &   
  DESC TODO \\ \hline
  
  TRIGGER\_LVL1 &   
  dataWidth &   
  DESC TODO \\ \hline
  
  TRIGGER\_STATUS &   
  dataWidth &   
  DESC TODO \\ \hline
  
  IRQ\_ENABLE &   
  dataWidth &   
  DESC TODO \\ \hline

\end{longtable}
